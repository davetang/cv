% with help from this awesome post
% http://texblog.org/2012/04/25/writing-a-cv-in-latex/

\documentclass[a4paper, 10pt]{article}
\usepackage[margin=3cm]{geometry}
\usepackage{helvet}

\usepackage{hyperref}
\hypersetup{
   colorlinks,
   citecolor=black,
   filecolor=black,
   linkcolor=black,
   urlcolor=blue
}

\usepackage{url}
\usepackage{bibentry}

% Reformat the date
% http://www.howtotex.com/packages/customize-the-date-format-in-your-latex-documents/
\usepackage{datetime}
\newdateformat{mydate}{\THEYEAR\ \monthname[\THEMONTH] \THEDAY\textsuperscript{th}}

\title{\bfseries\Huge Dave Ting Pong Tang}
\author{
   \href{mailto:davetingpongtang@gmail.com}{davetingpongtang@gmail.com}
}
\date{\mydate\today}

\usepackage{array, xcolor}
\definecolor{lightgray}{gray}{0.8}
\newcolumntype{L}{>{\raggedleft}p{0.14\textwidth}}
\newcolumntype{R}{p{0.8\textwidth}}
\newcommand\VRule{\color{lightgray}\vrule width 0.5pt}

\begin{document}

\maketitle

% http://en.wikibooks.org/wiki/LaTeX/Boxes#minipage_and_parbox
\begin{minipage}[ht]{.40\textwidth}\centering
100 Roberts Road, Subiaco \\
Western Australia 6008 \\
Australia
\end{minipage}
\begin{minipage}[ht]{.50\textwidth}\centering
Twitter: \href{https://twitter.com/davetang31}{@davetang31} \\
GitHub: \href{https://github.com/davetang}{davetang} \\
Blog: \href{http://davetang.org/muse}{http://davetang.org/muse}
\end{minipage}

\section*{Summary}

I am currently a post-doctoral researcher at the Telethon Kids Institute, working on the analysis of human genetic variants with respect to rare diseases. The main goal of my current research is to develop integrative approaches towards identifying disease-causing variants. Prior to this position, I was a Marie Curie Early Stage Researcher in the lab of Piero Carninci in RIKEN Yokohama and was primarily working on the analysis of high-throughput transcriptome sequencing data sets. During my time in RIKEN, I developed methods to identify sequencing artefacts and studied various classes of non-coding RNAs. I am passionate about bioinformatics and maintain a \href{http://davetang.org/muse}{technical blog} dedicated to bioinformatics. I am an open science and reproducible research advocate. It is my dream that one day my work will have a direct positive impact on the lives of others.

\section*{Personal}
\begin{tabular}{L!{\VRule}R}
   DOB & 1983 March 31\textsuperscript{st} \\
   Birth place & Hong Kong \\
   Nationality & Australian and British National (Overseas) \\
\end{tabular}

\section*{Education}
\begin{tabular}{L!{\VRule}R}
   2010--2015 & PhD Candidate, Vrije University, the Netherlands. \href{https://github.com/davetang/thesis}{PhD thesis}: High-throughput sequencing and transcriptomics: methods development and data analysis of large expression data sets. \\
   2001--2005 & BSc (Honours) in biochemistry and microbiology, University of Queensland, Australia. \href{https://github.com/davetang/cv/blob/master/honours_thesis.pdf}{Honours thesis}: Using a supertree approach to detect laterally transferred genes within \textit{Staphylococcus}. \\
\end{tabular}

\section*{Past Scientific Positions}
\begin{tabular}{L!{\VRule}R}
   2010--2015 & Research Associate in the lab of Piero Carninci at RIKEN Yokohama, Japan \\
   2008--2010 & Research Assistant in the lab of Sean Grimmond at The University of Queensland, Australia \\
   2006--2008 & Research Assistant in the lab of Brian Dalrymple at the Commonwealth Scientific and Industrial Research Organisation, Australia \\
   2005--2006 & Research Assistant in the lab of Mark Ragan at the University of Queensland, Australia \\
\end{tabular}

\section*{Research Interests}

\begin{itemize}
   \setlength\itemsep{0em}
   \item Investigating the potential role of genetic variants in relation to biological function and disease.
   \item Genomics and transcriptomics; in particular the study of non-coding RNAs and transposable elements.
   \item The application of bioinformatics, in particular \href{https://github.com/davetang/machine_learning}{machine learning} and data mining methods, to biological problems.
\end{itemize}

\section*{Bioinformatic Skills}

\begin{itemize}
   \setlength\itemsep{0em}
   \item Data analysis of high-throughput sequencing data from DNA-seq, RNA-seq, CAGE-seq, sRNA-seq, and ChIP-seq.
   \item Knowledge and the ability to use various bioinformatic databases, APIs, repositories, and tools.
   \item The application of biostatistics for the analysis of high-throughput sequencing data.
\end{itemize}

\section*{Computer Skills}

\begin{itemize}
   \setlength\itemsep{0em}
   \item Operating systems: Linux/Unix (RHEL/CentOS and Ubuntu), OS X, and Windows.
   \item Programming/scripting languages: Perl, R, \href{https://github.com/davetang/getting_started_with_c}{C}, Bash, JavaScript, HTML, SQL, and PHP.
   \item Open science and reproducible research tools: \href{https://github.com/davetang/getting_started_with_git}{git}, cloud computing (AWS), WordPress, Jekyll, \href{https://github.com/davetang/learning_docker/}{Docker}, and R Markdown/Markdown.
\end{itemize}

\section*{Honours and Awards}
\begin{tabular}{L!{\VRule}R}
   2010 & CSIRO Chairman's Medal \\
   2008 & CSIRO Partnership Excellence Award \\
\end{tabular}

\section*{Grants}
\begin{tabular}{L!{\VRule}R}
   2016 & \href{https://www.cancerwa.asn.au/research/funding/collaborative_cancer_grant_scheme/}{Cancer Council WA Collaborative Cancer Grant Scheme} CIC \\
   2015 & Telethon - Perth Children's Hospital Research Fund ``\href{http://telethonkids.org.au/our-research/projects-index/g/genetics-seqnextgen-translating-nextgen-sequencing-for-the-diagnosis-of-developmental-anomalies-and-rare-diseases/}{SeqNextGen}: Translating NextGen Sequencing for the Diagnosis of Developmental Anomalies and Rare Diseases'' AI4 AUD 192,505 \\
\end{tabular}

\section*{Academic and Administrative Experience}
\begin{tabular}{L!{\VRule}R}
   2015 & MODHEP workshop presenter on \href{https://github.com/davetang/cage_r}{analysing CAGE data} \\
   2013--2014 & Organiser of the Chat with Guest sessions at RIKEN CLST DGT \\
   2013--2014 & Organiser of the Student Journal Club at RIKEN CLST DGT \\
   2013 & Organising committee for the \href{http://www.nature.com/natureevents/science/events/20919-BrainTrain_Conference}{BrainTrain conference} \\
   2013 & Organiser of the BrainTrain workshop: Identifying regulatory elements in the genome \\
   2012 & Session chair for the \href{http://www.osc.riken.jp/english/event/2012/121115}{Patients and Medicines forum} \\
   2012 & Presenter at the RIKEN OSC bioinformatics course
\end{tabular}

\section*{Editorial Activities}
Referee/Reviewer (number of grant proposals/manuscripts reviewed in parenthesis) for: National Health and Medical Research Council (1), BMC Genomics (1), PeerJ (1), and Scientific Reports (2).

\section*{Workshop and Course Attendances}
\begin{tabular}{L!{\VRule}R}
   2016 & \href{https://www.australiangenomics.org.au/news-events/events/2016/agha-workshop-2016-reducing-morbidity-and-mortality-from-genetic-disease-through-screening/}{AGHA Workshop: reducing morbidity and mortality from genetic disease through screening} \\
   2016 & \href{http://www.involvingpeopleinresearch.org.au/}{Consumer and Community Involvement in Research} workshop \\
   2015 & UQ winter school in mathematical and computational biology \\
   2014 & RIKEN/KI doctoral course: Employing Genome-wide Technologies for Functional Regulation in Development and Disease \\
   2013 & AMATA conference ECR workshop \\
   2013 & Coursera data analysis course from Johns Hopkins University \\
   2013 & BrainTrain \href{http://www.brain-train.nl/training-2/}{courses and workshops} \\
   2013 & RIKEN/KI doctoral course: Epigenomics: Methods and Applications to Disease and Development \\
   2012 & SISSA summer school on dopaminergic neurons \\
   2012 & RIKEN/KI doctoral course: Functional Regulation in Development and Disease \\
   2011 & UQ winter school in mathematical and computational biology \\
   2011 & RIKEN/EBI bioinformatics roadshow \\
\end{tabular}

\section*{Hobbies and Interests}
Sports (especially basketball), cycling, bioinformatics blogging, self study, and reading.

\section*{Academic References}

\begin{minipage}[ht]{.50\textwidth}
Brian Dalrymple \\
Honorary Research Fellow, \\
Institute of Agriculture, \\
The University of Western Australia \\
Australia \\
\href{mailto:brian.dalrymple@uwa.edu.au}{brian.dalrymple@uwa.edu.au} \\
Phone: +61 408 697 130
\end{minipage}
\begin{minipage}[ht]{.50\textwidth}
Piero Carninci \\
RIKEN Yokohama Campus \\
1-7-22 Suehiro-cho, Tsurumi-ku, Yokohama \\
Kanagawa 230-0045 \\
Japan \\
\href{mailto:carninci@riken.jp}{carninci@riken.jp} \\
Phone: +81 45 503 9222
\end{minipage}

\section*{Publications}
% http://tex.stackexchange.com/questions/22645/hiding-the-title-of-the-bibliography
\begingroup
   \renewcommand{\section}[2]{}%
   \bibliographystyle{unsrturl}
   \nocite{*}
   \bibliography{pub}
\endgroup

\vfill

\footnotesize
This C.V. was prepared in \LaTeX\ and is available at \href{https://github.com/davetang/cv}{https://github.com/davetang/cv}.

\end{document}
